\section{Servidor de backend}

En esta sección se irán enumerando los cambios realizados sobre el juego original para cumplir con los requisitos dados.

Para permitir que múltiples clientes se conecten al backend de forma simultánea editamos el $main$ de $backend.cpp$ hicimos que no se cierre el socket del servidor al conectarse un usuario, sino que este se mantenga abierto para permitir que nuevos jugadores se conecten. Ahora cada vez que alguien se conecta se crea un nuevo $thread$ que ejecuta, al igual que en la versión mono, a $atendedor\_de\_jugador$.

Como desición de diseñó preferimos modelar el tablero del juego a partir de una sola matriz $tablero$ en vez de un $tablero\_temporal$ y $tablero\_confirmado$. De esta manera se permite para una casilla tener el RESERVADO, definido como 0x01 en $Encabezado.h$, esto indica que un jugador escogió poner una carta sobre la casilla pero aún no cofirmo la jugada. Haciendo esto pudimos luego modelar facilmente la restricción de un jugador para escoger una casilla reservada.

Otro pequeño cambio fue que las variables $socketfd_cliente$ y $socket_size$ se cambiaron de $int$ a $long$ ya que el primero, a diferencia del segundo, es un tipo de datos que es representado con un tamaño distinto dependiendo el sistema operativo, lo cual puede generar problemas y confusiones al intentar acceder al espacio de memoria que este ocupa.

Se agregó $tablero\_lock$ en $backend.cpp$, la variable local que utilizaremos para estar manejar las escrituras y lecturas del tablero, evitando condiciones de carrera y obteniendo un manejo sincronizado del tablero.

De esta manera se agregaron locks y unlocks donde fue necesario, mas especificamente en los siguientes métodos

\begin{itemize}

\item $atendedor\_de\_jugador$

\begin{itemize}
\item Se bloquea la escritura y lectura. Para agregar una nueva ficha a una jugada, reservando de esta manera una posición del tablero.

\item Se bloquea la escritura y lectura para escribir sobre el tablero la jugada confirmada.

\end{itemize}

\item $enviar\_tablero$

\begin{itemize}

\item Se bloquea la escritura. Para no enviar un tablero con una jugada a medio hacer.

\end{itemize}

\item $quitar\_cartas$

\begin{itemize}

\item Se bloquea la escritura y lectura. Para eliminar cartas del tablero de forma completa y evitar que un usuario pueda leer o escribir en un tablero con una jugada a medio quitar.

\end{itemize}

\end{itemize}
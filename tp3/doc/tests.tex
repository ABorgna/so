\section{Tests}

Para comprobar el correcto funcionamiento de las funciones anteriores vamos a hacer un seguimiento de un conjunto de operaciones específicadas en \emph{t1.txt} sobre 4 nodos.

\begin{itemize}
    \item \textbf{j 1:} Llega el nodo con rank 1, que pasará a ser el nodo de contacto inicial.
        Se inicializa con hash:

        48635463943209834798109814161294753926839975257569795305637098542720658922315.

    \item \textbf{s 1 a1:} Llega el archivo a1 con hash:

        110986164698320180327942133831752629430491002266485370052238869825166557303060,
        como el nodo 1 es el único, solo él lo guarda.

    \item \textbf{s 1 a2:} Al igual que antes, el nodo 1 almacena el archivo a2 con hash:

        20004699973444259106832769305100157129890497967822946728731731367322388695815.

    \item \textbf{j 2:} Llega el nodo 2 a la red, con hash:

        96094161643976066833367867971426158458230048495430276217795328666133331159861.

        Se comunica con el nodo de contacto 1 por nodos cercanos, este no encuentra
        otros nodos (más que el mismo) para enviar y no encuentra tampoco archivos con
        menor distancia al hash del nodo 2 que él mismo:

        distance(hash\_fn("1"),hash\_fn(''a1")) = 119 < 122 = distance(hash\_fn("2"),hash\_fn(''a1"))

        y lo mismo para a2:

        distance(hash\_fn("1"),hash\_fn(''a2")) = 118 < 127 = distance(hash\_fn("2"),hash\_fn(''a2"))).

    \item \textbf{j 3:} Llega el nodo 3 a la red, con hash:

        35293215426786447154857697798367884701614677727176325092965345248689205321678

        Se comunica con el nodo de contacto 1, este lo agrega a su propia tabla de ruteo
        y este le envía el contacto del nodo 2 el cual, al recibir el request de join, lo agrega a tu tabla de ruteo sin enviarle archivos dado que su tabla de archivos es vacía.

        Como ambos archivos quedan a menor distancia del nodo 1 que del 3, no se produce
        migración de estos.

        Para a1:

        distance(hash\_fn("1"),hash\_fn(''a1")) = 119 < 136 = distance(hash\_fn("3"),hash\_fn(''a1"))).

        Y para a2:

        distance(hash\_fn("1"),hash\_fn(''a2")) = 118 < 133 = distance(hash\_fn("3"),hash\_fn(''a2"))).

    \item \textbf{l 3 a1:} Leemos el archivo a1 desde el nodo 3

        El nodo 3 busca el nodo con menor distancia a a1 entre los de su tabla de ruteo,
        selecciona al nodo 1, y le envía el hash del archivo.

        El nodo 1 encuentra al archivo a1 entre los que tiene almacenados, y se lo envía al
        nodo 3.

    \item \textbf{s 3 a3:} Llega el archivo a3 al nodo 3 con hash:

        110558374480492830394381912301702732669425904788263259158895862965223613393983

        con distancias

        distance(hash\_fn("1"),hash\_fn(''a3")) = 134

        distance(hash\_fn("2"),hash\_fn(''a3")) = 109

        distance(hash\_fn("3"),hash\_fn(''a3")) = 127

        El nodo 3 busca el nodo con menor distancia al archivo entre los de su tabla,
        y se lo envía al nodo 2.

        Como el nodo 2 no conoce ningún otro nodo mas cercano al archivo a3, lo almacena.

    \item \textbf{j 4:} Llega el nodo 4 a la red, con hash:

        33984360982413536682390860969296307922929415152052354251133793603654468157322

        Se comunica con el nodo de contacto 1, este lo agrega a su propia tabla de ruteo.
        Luego busca entre los nodos que conoce al mas cercano al nodo 4 y le envía al nodo
        2 la el hash de este. El nodo 2 no conoce a ningún otro nodo mas cercano, por lo que
        solo lo agrega a su tabla de ruteo. Finalmente, el nodo 1 responde al nodo 4
        con el hash del nodo 2, para que este lo agregue a su tabla de ruteo.

        Como todos los archivos están a una distancia mayor del nodo 4 que de su nodo actual,
        no hay migraciones.

        distance(hash\_fn("1"),hash\_fn(''a1")) = 119 < 121 = distance(hash\_fn("4"),hash\_fn(''a1"))).
        distance(hash\_fn("1"),hash\_fn(''a2")) = 118 < 132 = distance(hash\_fn("4"),hash\_fn(''a2"))).
        distance(hash\_fn("2"),hash\_fn(''a3")) = 109 < 140 = distance(hash\_fn("4"),hash\_fn(''a3"))).

    \item \textbf{l 4 a2:} Leemos el archivo a2 desde el nodo 4

        El nodo 4 busca el nodo con menor distancia a a2 entre los de su tabla de ruteo,
        selecciona al nodo 1, y le envía el hash del archivo.

        El nodo 1 encuentra al archivo a2 entre los que tiene almacenados, y se lo envía al
        nodo 4.

    \item \textbf{l 4 a3:} Leemos el archivo a3 desde el nodo 4

        El nodo 4 busca el nodo con menor distancia a a3 entre los de su tabla de ruteo,
        selecciona al nodo 2, y le envía el hash del archivo.

        El nodo 2 encuentra al archivo a3 entre los que tiene almacenados, y se lo envía al
        nodo 4.

\end{itemize}

Se puede ver, al correr la versión bloqueante, de que el log al ejecutar el test presenta el mismo comportamiento especificado anteriormente y las funciones que implementamos se comportaron como esperábamos. En el caso de la versión no bloqueante, debido al comportamiento no determinístico (consecuencia de que al propagar consultas a través de nodos de contacto recibimos las respuestas sin un orden específico) podría variar el resultado pero, en casi todos los casos que probamos el comportamiento fue el mismo.

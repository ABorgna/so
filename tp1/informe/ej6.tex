\section{Ejercicio 6}

\begin{enumerate}[label=\alph*)]

    \item


        /*--------------------------------*

        Parrafo del paper:

        In this paper, we both focus on resource allocation and task
scheduling, as well as take into account some specif
c criteria
or priorities of tasks and resources, such as resource bandwidth
and processing capability, task granularity (f
ne-grained and
coarse-grained) and deadline.

(Habría que decir que los "recursos" (nucleos?) no son homogeneos,
pueden tener distintas características)

        *--------------------------------*/

    \item El algoritmo se ocupa de seleccionar tanto
        la siguiente tarea a ejecutar como el recurso donde correrla.

        Los diseños clásicos basados en agrupamientos juntan tareas de tipo
        y características similares en lotes a ser despachados conjuntamente
        a un recurso afin a las prioridades de las tareas.
        El tamaño del grupo queda determinado por la cantidad de operaciones
        estimada que tomará, la cuál no debe superar el límite de operaciones
        asignado al recurso objetivo.

        El algoritmo propuesto agrega una condición cuando el tiempo requerido
        por una tarea es mayor al de los recursos necesarios, y la despacha
        en un grupo separado para si sola. Ademas, intenta siempre ocupar
        el recurso con mayor velocidad de ejecución primero, para así minimizar
        el waiting time.

        TODO: hace falta escribir el algoritmo en si?

    \item

\end{enumerate}


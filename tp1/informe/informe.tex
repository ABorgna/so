\documentclass[hidelinks,a4paper,12pt, nofootinbib]{article}
\usepackage[nounderscore]{syntax}
\usepackage[left=3cm, top=2.5cm, right=2cm, left=2.5cm]{geometry}
\usepackage[spanish, es-tabla]{babel} %es-tabla es para que ponga Tabla en vez de Cuadro en el caption
\usepackage[utf8]{inputenc}
\usepackage[T1]{fontenc}
\usepackage{xspace}
\usepackage{xargs}
\usepackage{fancyhdr}
\usepackage{lastpage}
\usepackage{caratula}
\usepackage[bottom]{footmisc}
\usepackage{amsmath}
\usepackage{amssymb}
\usepackage{array}
\usepackage{xcolor,colortbl}
\usepackage{amsthm}
\usepackage{listings}
\usepackage{soul}
\usepackage[titletoc]{appendix}
\usepackage{pgf}
\usepackage{graphicx}
\usepackage{sidecap}
\usepackage{amsmath}
\usepackage{wrapfig}
\usepackage{caption}
\usepackage{subcaption}


%\usepackage{minted}

%Formato de los links
\usepackage{hyperref}
\hypersetup{
  colorlinks   = true, %Colours links instead of ugly boxes
  urlcolor     = blue, %Colour for external hyperlinks
  linkcolor    = blue, %Colour of internal links
  citecolor   = red %Colour of citations
}

\usepackage{comment}

\captionsetup[table]{labelsep=space}


\setlength{\parindent}{4em}
\setlength{\parskip}{0.5em}


%%fancyhdr
\pagestyle{fancy}
\thispagestyle{fancy}
\addtolength{\headheight}{1pt}
\lhead{Teoría de Lenguajes: TP1}
\rhead{$1º$ cuatrimestre de 2016}
\cfoot{\thepage\ / \pageref{LastPage}}
\renewcommand{\footrulewidth}{0.4pt}
\renewcommand{\labelitemi}{$\bullet$}
\setcounter{section}{-1}

%%caratula
\materia{Sistemas Operativos}
\titulo{Trabajo Práctico Número 1}
%\subtitulo{}
\grupo{}
\integrante{Borgna, Agustín}{79/15}{aborgna@dc.uba.ar}
\integrante{Gatti, Mathias}{477/14}{mathigatti@gmail.com}
\integrante{Lancioni, Franco}{234/15}{glancioni@dc.uba.ar}

\fecha{7 de septiembre de 2016}

\usepackage{etoolbox}
\AtBeginEnvironment{tikzpicture}{\shorthandoff{>}\shorthandoff{<}}{}{}
\newcommand{\cupdot}{\mathbin{\mathaccent\cdot\cup}}
\title{tp-leng}
\begin{document}
\maketitle

\tableofcontents
\newpage


\section{Ejercicio 1}


\section{Ejercicio 2}


\section{Ejercicio 3}

La \textbf{TaskBatch} debe usar la CPU durante $total\_cpu$ ciclos de reloj.
Como cada bloqueo utiliza un ciclo para iniciarse, el tiempo extra que tendremos que
usar la CPU sera de $t_{total} = total\_cpu - cant\_bloqueos$.

Para acomodar todos los bloqueos pseudo-aleatoriamente decidimos en qué luego de cuanto
tiempo extra lanzar cada una utilizando la función \textsf{rand()} y la insertamos en un vector ordenado.
Luego procedemos recorremos el array esperando el tiempo necesario.

El algoritmo se detalla en el listing \ref{ej3-algo}

\begin{lstlisting} [caption={Generacion aleatoria de la secuencia de bloqueos},label=ej3-algo]j
for $i$ in range($cant\_bloqueos$):
    agregar (rand() % ($t_{total} + 1$)) al array ordenado $bloqueos$

$tiempoExtraConsumido$ $\leftarrow$ 0
for $t_{bloqueo}$ in $bloqueos$:
    if $tiempoExtraConsumido$ < $t_{bloqueo}$:
        usar la cpu por ($t_{bloqueo}$ - $tiempoExtraConsumido$) ciclos
        $tiempoExtraConsumido$ $\leftarrow$ $t_{bloqueo}$
    bloquear la tarea por 2 ciclos

if $tiempoExtraConsumido$ < $t_{total}$:
    usar la cpu por ($t_{total}$ - $tiempoExtraConsumido$) ciclos

\end{lstlisting}

Utilizando un lote de \textbf{TaskBatch} con parámetros detallados en la tabla \ref{table:ej3-lote}
podemos ver en la figura \ref{fig:ej3} como se logra el comportamiento deseado.

\begin{center}
    \begin{tabular}{| l | l | l |}
        \hline
        Tarea & total\_cpu & cant\_bloqueos \\ \hline
        0 & 10 & 4 \\
        1 & 15 & 4 \\
        2 & 10 & 2 \\
        3 & 15 & 6 \\
        \hline
    \end{tabular}
    \label{table:ej3-lote}
\end{center}

\begin{figure}[H]
    \centering
    \includegraphics[width=\textwidth]{ej3}
    \caption{Lote de 4 TaskBatch}
    \label{fig:ej3}
\end{figure}


\section{Ejercicio 4}

\label{sec:rr}

Para la implementación de \textbf{SchedRR} usamos una única cola de procesos para todos los cores, un vector que indica para cada cpu su cantidad de quantums y otro del mismo tamaño para indicar el estado actual de cada uno.
  \begin{lstlisting} [caption={Estructuras de la clase}]
     std::queue<int> q;
     std::vector<int> quantum_total;
     std::vector<int> quantum_restante;
  \end{lstlisting}

De este modo, la idea es en cada $TICK$ decrementar los quantums restantes del cpu correspondiente y, cuando llegue a cero habiendo otras tareas en la cola, desencolar una nueva.

Cuando se introduce un nuevo proceso u otro vuelve de una llamada bloqueante lo único que debemos hacer es agregarlo a la cola de espera del scheduler.
En el caso de un $tick$, la acción difiere según el motivo con el que se lo llama:

  \begin{itemize}
    \item Si se trata de un $BLOCK$ y hay procesos esperando ser ejecutados, se desencola uno sin encolar el actual (en su lugar, esperamos a que llame a $unblock$ para volver a la cola). De no haber más procesos, se le sigue destinando tiempo de cpu al actual mientras esté bloqueado.
    \item De ser un $EXIT$, significa que la tarea actual ha concluído su ejecución. Al igual que en el block, desencolamos un nuevo proceso y no reencolamos la actual porque ya terminó de ejecutar. De no haber, sin embargo, pasamos a la $IDLE\_TASK$ hasta que aparezca una nueva tarea y el cpu reciba un $tick$.
    \item Por otro lado, cuando el motivo es un $TICK$, decrementamos siempre los quantums restantes del cpu actual y, mientras haya, seguirá ejecutando su tarea. Si la tarea actual se quedó sin quantums o bien la reencolamos tras sacar una nueva con el total de sus quantums, o renovamos su ciclo y la dejamos seguir ejecutándose.
  \end{itemize}

En esta implementación, de cargar una tarea justo después de que la única en ejecución termine su ciclo, la tarea nueva va a tener que esperar a que la actual termine otro ciclo más.

  \begin{lstlisting} [caption={Algoritmo para operaci\'on tick en round robin}]
    if TICK:
      decrementar quantum[cpu]
      if quantum[cpu] > 0:
        return pid
      if no hay procesos encolados y pid $\neq$ IDLE:
        resetear quantum[cpu]
        return pid
      if pid $\neq$ IDLE:
        encolar pid

    if no hay procesos encolados:
      if BLOCK return pid
      else return IDLE

    return nuevo proceso desencolado y resetear quantum[cpu]
  \end{lstlisting}

\section{Ejercicio 5}


\section{Ejercicio 6}

\begin{enumerate}[label=\alph*)]

    \item

	El paper se centra en la asignación de recursos y el scheduling de tareas considerando criterios y prioridades específicas para las diferentes características de las tareas solicitadas y recursos disponibles. Estas características son el ancho de banda de un recurso en particular, el cual afecta la latencia de las comunicaciones y las tasas de transmisión;
la capacidad de procesamiento del recurso, medido en $MIPS$ (Million of instructions per second); o la granularidad de cada tarea, es decir, la proporción de comunicación sobre cómputo que realizan. Aquellas que toman poco tiempo en procesar y comunican datos con frecuencia se denominan \emph{fine-grained}, en contraposición con aquellas que requieren mayor tiempo de ejecución y menos sincronización llamadas \emph{coarse-grained}. 

	Una de las motivaciones para considerar la granularidad en sistemas en la nube es que si otorgamos recursos de alta capacidad de cómputo para tareas \emph{fine-grained}, estaríamos provocando overhead innecesario por destinar esos recursos para tareas que gastan mucho tiempo en scheduling y transmisión cuando sería óptimo que ejecuten siempre que sea posible.
	
	Otra característica a tener en cuenta es el \emph{deadline} de las tareas demandadas por clientes. Una de las principales motivaciones para la implementación propuesta en el paper además de minimizar waiting-time y desperdicios de recursos es, justamente, mejorar la calidad de servicio (\emph{QoS, Quality of Service}) percibida por el cliente.

    \item El algoritmo generado para resolver el problema propuesto en el articulo es llamado $task grouping
algorithm integrated with SJF and bandwidth awareness$, este se ocupa de seleccionar tanto
        la siguiente tarea a ejecutar como el recurso donde correrla.

        Los diseños clásicos basados en agrupamientos juntan tareas de tipo
        y características similares en lotes a ser despachados conjuntamente
        a un recurso afin a las prioridades de las tareas.
        El tamaño del grupo queda determinado por la cantidad de operaciones
        estimada que tomará, la cuál no debe superar el límite de operaciones
        asignado al recurso objetivo.

        El algoritmo propuesto agrega una condición cuando el tiempo requerido
        por una tarea es mayor al de los recursos necesarios, y la despacha
        en un grupo separado para si sola. Ademas, intenta siempre ocupar
        el recurso con mayor velocidad de ejecución primero, para así minimizar
        el waiting time.

    \item La experimentación consiste en comparar la performance del algoritmo de scheduling propuesto frente a otras posibles opciones. Para medir el desempeno se corrén los schedulers sobre un simulador de infraestructuras y servicios de cloud computing llamado CloudSim. Para cada scheduler se medirá el tiempo de espera medio entre todos los procesos, el tiempo total de procesamiento (Turnaround) y el costo de procesamiento variando la cantidad de procesos posibles y el tamaño de la granularidad. Para simular el ambiente de trabajo de un scheduler se generan tareas las cuales varian en sus requerimientos de procesado y tamaño de forma aleatoria.

El algoritmo del articulo, en comparación con los algoritmos de task grouping existentes, muestra una disminución en el waiting time y el processing time de mas del 30\%. Estos resultados favorables luego de la experimentación pudieron corroborar que el método propuesto, el algoritmo de task grouping integrado con SJF y bandwidth awareness, minimiza efectivamente el waiting time y el processing time al mismo tiempo que reduce el processing cost logrando una utilización optima de los recursos y un minimo overhead, reduciendo al mismo tiempo bottlenecks en el ancho de banda en comunicacion.

\end{enumerate}


\section{Ejercicio 7}

\subsection{Shortest Job First - No reentrante}

  La implementación del scheduler de tipo \textbf{SchedSJF} ($Shortest\ Job\ First$) es muy similar a la de \textbf{SchedFCFS} porque ninguno es reentrante. La diferencia es que en vez de usar una cola común, una cola de prioridad respecto de las tareas con menor tiempo de ejecución total.
  Además, por enunciado, tenemos asegurado que solamente correrán tareas de tipo $TaskCPU$. Es decir, no existen bloqueos y, por lo tanto, el tiempo de ejecución de cada tarea (pasado por parámetro) coincide exactamente con la cantidad de ciclos de cpu que va a consumir.
  \\

  Al inicializar el scheduler insertamos en un $map$ con $pid's$ de claves sus respectivos tiempos totales de ejecución.

  \begin{lstlisting} [caption={Estructuras de la clase}]
    template<typename T>
    using minheap = priority_queue<T, vector<T>, std::greater<T>>;
    [...]

      minheap<pair<int, int>> espera;
      map<int, int> tiempos;
  \end{lstlisting}


  De esta manera, cada vez que se cargue un $pid$ en el scheduler vamos a buscar su correspondiente tiempo de ejecución en el $map$ y encolar el par <$tiempo \ de \ ejecucion$, $pid$> en la cola de prioridad. Como el tipo $pair$ resuelve la relación de orden '<' priorizando el primer elemento, podemos asegurar que $espera.top()$ retorna un par con menor o igual tiempo de ejecución que el resto del conjunto.
  \\

  Partiendo de esto, el resto del scheduler se comporta idénticamente a uno de tipo $FCFS$.

  \begin{lstlisting} [caption={Implementaci\'on de operaci\'on $tick$}]
    if EXIT y no hay mas procesos en espera:
      return IDLE
    else if TICK:
      if  tarea actual $\neq$ IDLE:
        return tarea actual
      else if no hay mas procesos en espera:
        return IDLE

    return pid desencolado de espera
  \end{lstlisting}


  Para ver un ejemplo, probemos con el siguiente caso simple:

  \begin{center}
  \fbox{\parbox{6cm}
  	{\noindent
      TaskCPU 10  \\
      TaskCPU 9   \\
      @4:         \\
      TaskCPU 2
  	}
  }
  \end{center}
  \captionof{figure}{Lote de tareas usado}
  \label{lst:ej7.lote}

  \begin{figure}[H]
  		\centering
  		\includegraphics[width=\textwidth]{ej7-sjf}
  		\caption{Ejecución del lote de la figura \ref{lst:ej7.lote} con un \'unico cpu }
  		\label{fig:ej7.sjf}
  \end{figure}

\subsection{Shortest Job First - Reentrante}

  La implementación del \textbf{SchedRSJF} usa exactamente la misma idea respecto del uso de un $map$ y una cola de prioridad para el tiempo de cada proceso. La diferencia es que, al ser reentrante, las tareas no se ejecutan necesariamente hasta que terminen. En cambio, se van rotando por turnos como en el \hyperref[sec:rr]{\textbf{SchedRR}} pero usando una cola de prioridad respecto de los procesos con menor tiempo restante de ejecución. Podemos confiar en que estos tiempos se ajusten a la práctica porque, al igual que antes, podemos asumir que solamente se ejecutarán tareas de tipo $TaskCPU$ y por lo tanto no habrá bloqueos.

  \begin{lstlisting} [caption={Estructuras de la clase}]
    minheap<pair<int, int>> espera;
    map<int, int> tiempos;

    vector<int> quantum_restante;
    vector<int> quantum_total;
  \end{lstlisting}

  Salvo por el agregado de que no hay $ticks$ con motivo $BLOCK$, la operación sigue la misma idea. La única diferencia radica en que en cada $TICK$ hay que descontar el tiempo restante de ejecución del $pid$ actual para luego actualizar el par < $tiempo$, $pid$ >  de la cola de prioridad al reencolar la tarea.

  \begin{lstlisting} [caption={Algoritmo para operaci\'on tick en shortest job first reentrante}]
    if TICK:
      decrementar quantum[cpu] y tiempos[pid]
      if quantum[cpu] > 0:
        return pid
      if no hay procesos encolados y pid $\neq$ IDLE:
        resetear quantum[cpu]
        return pid
      if pid $\neq$ IDLE:
        encolar <tiempos[pid], pid>

    if no hay procesos encolados:
      if BLOCK return pid
      else return IDLE

    return nuevo pid del par desencolado y resetear quantum[cpu]
  \end{lstlisting}

  \begin{figure}[H]
  		\centering
  		\includegraphics[width=\textwidth]{ej7-rsjf}
  		\caption{Ejecución del lote de la figura \ref{lst:ej7.lote} con un \'unico cpu de 4 quantums por turno}
  		\label{fig:ej7.rsjf}
  \end{figure}

  Se puede ver que, al terminar el segundo turno de la tarea $1$, se volvió a ejecutar esta misma tarea en vez de la tarea $0$ porque le quedaba menos tiempo restante de ejecución.

\section{Ejercicio 8}

Realizamos las mediciones sobre un lote de tareas $TaskCPU$,
ya que los schedulers SJF y RSJF solo soportan tareas de este tipo.

\begin{center}
\fbox{\parbox{6cm}
    {\noindent
        TaskCPU 10 \\
        TaskCPU 20 \\
        @4: \\
        TaskCPU 5 \\
        TaskCPU 5 \\
        TaskCPU 5
    }
}
\end{center}
\captionof{figure}{Lote de tareas}
\label{fig:ej8-t1}
\vspace*{1em}

\subsection{Mediciones sobre un solo núcleo}

Primero realizamos mediciones ejecutando sobre un solo núcleo.

\begin{figure}[H]
        \centering
        \includegraphics[width=\textwidth]{ej8-cpu-rr}
        \caption{Ejecución sobre un único core del lote de la figura
            \ref{fig:ej8-t1} con el scheduler RoundRobin}
        \label{fig:ej8-cpu-rr}
\end{figure}

\begin{figure}[H]
        \centering
        \includegraphics[width=\textwidth]{ej8-cpu-fcfs}
        \caption{Ejecución sobre un único core del lote de la figura
            \ref{fig:ej8-t1} con el scheduler FIFO}
        \label{fig:ej8-cpu-fcfs}
\end{figure}

\begin{figure}[H]
        \centering
        \includegraphics[width=\textwidth]{ej8-cpu-sjf}
        \caption{Ejecución sobre un único core del lote de la figura
            \ref{fig:ej8-t1} con el scheduler SJF}
        \label{fig:ej8-cpu-sjf}
\end{figure}

\begin{figure}[H]
        \centering
        \includegraphics[width=\textwidth]{ej8-cpu-rsjf}
        \caption{Ejecución sobre un único core del lote de la figura
            \ref{fig:ej8-t1} con el scheduler RSJF}
        \label{fig:ej8-cpu-rsjf}
\end{figure}

\begin{center}
        \begin{tabular}{| c | c | c | c | c | c | c | c | c | c | c | c | c |}
                \hline
    Task & \multicolumn{4}{c |}{Waiting-time} & \multicolumn{4}{c |}{Latencia} & \multicolumn{4}{c |}{Turnaround} \\
                \cline{2-13}
          & RR & FIFO & SJF & RSJF & RR & FIFO & SJF & RSJF & RR & FIFO & SJF & RIFO \\
                \hline
    0 &       39 &   1 &   1 &  20 &        1 &   1 &   1 &   1 &       44 &  11 &  11 &  30 \\
    1 &       37 &  12 &  30 &  35 &        6 &  12 &  30 &  31 &       57 &  32 &  50 &  55 \\
    2 &       28 &  29 &   8 &   2 &        7 &  29 &   8 &   2 &       33 &  34 &  13 &   7 \\
    3 &       30 &  35 &  14 &   8 &       12 &  35 &  14 &   8 &       35 &  40 &  19 &  13 \\
    4 &       32 &  46 &  20 &  14 &       17 &  46 &  20 &  14 &       37 &  46 &  25 &  19 \\
                \hline
        \end{tabular}
\end{center}
\captionof{table}{Mediciones sobre un solo core
                  usando el lote de tareas de la figura \ref{fig:ej8-t1}}

Con estos valores podemos calcular el promedio de cada métrica para cada scheduler.

\begin{center}
        \begin{tabular}{| l | c | c | c |}
                \hline
    Scheduler & Waiting-time & Latencia & Turnaround \\
                \hline
    RR   &    33.5 &  8.6 & 41.2 \\
    FIFO &    24.6 & 24.6 & 32.6 \\
    SJF  &    14.6 & 14.6 & 23.6 \\
    RSJF &    15.8 & 11.2 & 24.8 \\
                \hline
        \end{tabular}
\end{center}
\captionof{table}{Promedio de métricas sobre un solo core
                  usando el lote de tareas de la figura \ref{fig:ej8-t1}}

\subsection{Mediciones sobre dos núcleos}

\begin{figure}[H]
        \centering
        \includegraphics[width=\textwidth]{ej8-cpu-rr-m}
        \caption{Ejecución sobre dos cores del lote de la figura
            \ref{fig:ej8-t1} con el scheduler RoundRobin}
        \label{fig:ej8-cpu-rr-m}
\end{figure}

\begin{figure}[H]
        \centering
        \includegraphics[width=\textwidth]{ej8-cpu-fcfs-m}
        \caption{Ejecución sobre dos cores del lote de la figura
            \ref{fig:ej8-t1} con el scheduler FIFO}
        \label{fig:ej8-cpu-fcfs-m}
\end{figure}

\begin{figure}[H]
        \centering
        \includegraphics[width=\textwidth]{ej8-cpu-sjf-m}
        \caption{Ejecución sobre dos cores del lote de la figura
            \ref{fig:ej8-t1} con el scheduler SJF}
        \label{fig:ej8-cpu-sjf-m}
\end{figure}

\begin{figure}[H]
        \centering
        \includegraphics[width=\textwidth]{ej8-cpu-rsjf-m}
        \caption{Ejecución sobre dos cores del lote de la figura
            \ref{fig:ej8-t1} con el scheduler RSJF}
        \label{fig:ej8-cpu-rsjf-m}
\end{figure}

\begin{center}
        \begin{tabular}{| c | c | c | c | c | c | c | c | c | c | c | c | c |}
                \hline
    Task & \multicolumn{4}{c |}{Waiting-time} & \multicolumn{4}{c |}{Latencia} & \multicolumn{4}{c |}{Turnaround} \\
                \cline{2-13}
          & RR & FIFO & SJF & RSJF & RR & FIFO & SJF & RSJF & RR & FIFO & SJF & RIFO \\
                \hline
    0 &       16 &   1 &   1 &  10 &        1 &   1 &   1 &   1 &       26 &  11 &  11 &  20 \\
    1 &       17 &   1 &   1 &  16 &        1 &   1 &   1 &   1 &       37 &  21 &  21 &  36 \\
    2 &       12 &   8 &   8 &   2 &        2 &   8 &   8 &   2 &       17 &  13 &  13 &   7 \\
    3 &       14 &  14 &  14 &   2 &        2 &  14 &  14 &   2 &       19 &  19 &  19 &   7 \\
    4 &       15 &  18 &  18 &   8 &        7 &  18 &  18 &   8 &       20 &  23 &  23 &  13 \\
                \hline
        \end{tabular}
\end{center}
\captionof{table}{Mediciones sobre dos cores
                  usando el lote de tareas de la figura \ref{fig:ej8-t1}}

Con estos valores podemos calcular el promedio de cada métrica para cada scheduler.

\begin{center}
        \begin{tabular}{| l | c | c | c |}
                \hline
    Scheduler & Waiting-time & Latencia & Turnaround \\
                \hline
    RR   &    14.8 &  2.6 & 23.8 \\
    FIFO &     8.4 &  8.4 & 17.4 \\
    SJF  &     8.4 &  8.4 & 17.4 \\
    RSJF &     7.6 &  2.8 & 16.6 \\
                \hline
        \end{tabular}
\end{center}
\captionof{table}{Promedio de métricas sobre dos cores
                  usando el lote de tareas de la figura \ref{fig:ej8-t1}}

\subsection{Conclusiones}

TODO


\section{Ejercicio 9}

\begin{figure}[H]
		\centering
		\includegraphics[width=\textwidth]{ej9}
		\caption{asdae}
		\label{fig:ej9}
\end{figure}



\end{document}

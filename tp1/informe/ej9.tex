\section{Ejercicio 9}

Realizamos algunas pruebas del scheduler MFQ a partir de dos simples ejemplos. El primero recibe 4 tareas simultaneamente en tiempo 0 y el segundo las va recibiendo de a una por vez a medida que el tiempo avanza.

\begin{figure}[H]
		\centering
		\includegraphics[width=\textwidth]{ej9-1}
		\caption{Ejercución sobre un core, las tareas 0 y 3 son de tipo TaskCPU, la 1 y 2 TaskConsola.}
		\label{fig:ej9}
\end{figure}

Al seguir el diagrama de Gantt del primer ejemplo se puede ver como los bloqueos de las tareas 1 y 2 le dan prioridad a dichos procesos, esta es una particularidad en el diseño del scheduler MFQ el cual toma la decisión de nunca bajar a una cola de menor nivel un proceso que se bloqueó. Al terminar las tareas 1 y 2, le toca el turno a la tarea 0 la cual habia ido a parar a una cola de menor nivel de prioridad. Esta cola tiene un quantum de 4, a diferencia de la de mayor prioridad que lo tiene de 2.

\begin{figure}[H]
		\centering
		\includegraphics[width=\textwidth]{ej9-2}
		\caption{Ejecución sobre dos cores, las tareas 0 y 3 son de tipo TaskCPU, la 1 y 2 TaskConsola. }
		\label{fig:ej9}
\end{figure}

En este ejemplo se sigue notando la prioridad que se le da a los procesos que se bloquean, aunque al haber dos cores los procesos 0 y 4 no necesitan esperar tanto para terminar, también se puede observar como al ir bajando a colas de menor prioridad los procesos 0 y 4 comienzan a durar mas tiempo corriendo, pasando de 2 ticks a 3 y luego 4.

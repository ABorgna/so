\section{Ejercicio 2}
\label{sec:ej2}

Podemos calcular la latencia de cada tarea como el tiempo desde que se carga hasta que comienza a ejecutarse y el throughput final como la cantidad de tareas ejecutadas dividida el tiempo total de ejecución.

\begin{figure}[H]
    \centering
    \includegraphics[width=\textwidth]{ej2-1}
    \caption{TaskConsola con n=4, bmin=1 y bmax=8}
    \label{fig:ej2.1}
\end{figure}

En la figura \ref{fig:ej2.1}, utilizando un solo nucleo, conseguimos un throughput de $\frac{4}{64} = 0.0625$.

Las latencias se detallan a continuación.

\begin{center}
    \begin{tabular}{| l | l |}
        \hline
        Tarea & Latencia \\ \hline
        0 & 2 \\
        1 & 9 \\
        2 & 30 \\
        3 & 46 \\
        \hline
    \end{tabular}
\end{center}

\begin{figure}[H]
    \centering
    \includegraphics[width=\textwidth]{ej2-2}
    \caption{TaskConsola con n=4, bmin=1 y bmax=8}
    \label{fig:ej2.2}
\end{figure}

En la figura \ref{fig:ej2.2}, utilizando 2 nucleos, conseguimos un throughput de $\frac{4}{39} = 0.1026$.

Las latencias se detallan a continuación.

\begin{center}
    \begin{tabular}{| l | l |}
        \hline
        Tarea & Latencia \\ \hline
        0 & 2 \\
        1 & 2 \\
        2 & 8 \\
        3 & 21 \\
        \hline
    \end{tabular}
\end{center}

\begin{figure}[H]
    \centering
    \includegraphics[width=\textwidth]{ej2-4}
    \caption{TaskConsola con n=4, bmin=1 y bmax=8}
    \label{fig:ej2.4}
\end{figure}

En la figura \ref{fig:ej2.4}, utilizando 4 nucleos, conseguimos un throughput de $\frac{4}{27} = 0.1481$.

Las latencias se detallan a continuación.

\begin{center}
    \begin{tabular}{| l | l |}
        \hline
        Tarea & Latencia \\ \hline
        0 & 2 \\
        1 & 2 \\
        2 & 2 \\
        3 & 2 \\
        \hline
    \end{tabular}
\end{center}

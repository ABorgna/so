\section{Ejercicio 8}

Realizamos las mediciones sobre un lote de tareas $TaskCPU$,
ya que los schedulers SJF y RSJF solo soportan tareas de este tipo.

\begin{center}
\fbox{\parbox{6cm}
    {\noindent
        TaskCPU 10 \\
        TaskCPU 20 \\
        @4: \\
        TaskCPU 5 \\
        TaskCPU 5 \\
        TaskCPU 5
    }
}
\end{center}
\captionof{figure}{Lote de tareas}
\label{fig:ej8-t1}
\vspace*{1em}

\subsection{Mediciones sobre un solo núcleo}

Primero realizamos mediciones ejecutando sobre un solo núcleo.

\begin{figure}[H]
        \centering
        \includegraphics[width=\textwidth]{ej8-cpu-rr}
        \caption{Ejecución sobre un único core del lote de la figura
            \ref{fig:ej8-t1} con el scheduler RoundRobin}
        \label{fig:ej8-cpu-rr}
\end{figure}

\begin{figure}[H]
        \centering
        \includegraphics[width=\textwidth]{ej8-cpu-fcfs}
        \caption{Ejecución sobre un único core del lote de la figura
            \ref{fig:ej8-t1} con el scheduler FIFO}
        \label{fig:ej8-cpu-fcfs}
\end{figure}

\begin{figure}[H]
        \centering
        \includegraphics[width=\textwidth]{ej8-cpu-sjf}
        \caption{Ejecución sobre un único core del lote de la figura
            \ref{fig:ej8-t1} con el scheduler SJF}
        \label{fig:ej8-cpu-sjf}
\end{figure}

\begin{figure}[H]
        \centering
        \includegraphics[width=\textwidth]{ej8-cpu-rsjf}
        \caption{Ejecución sobre un único core del lote de la figura
            \ref{fig:ej8-t1} con el scheduler RSJF}
        \label{fig:ej8-cpu-rsjf}
\end{figure}

\begin{center}
        \begin{tabular}{| c | c | c | c | c | c | c | c | c | c | c | c | c |}
                \hline
    Task & \multicolumn{4}{c |}{Waiting-time} & \multicolumn{4}{c |}{Latencia} & \multicolumn{4}{c |}{Turnaround} \\
                \cline{2-13}
          & RR & FIFO & SJF & RSJF & RR & FIFO & SJF & RSJF & RR & FIFO & SJF & RIFO \\
                \hline
    0 &       39 &   1 &   1 &  20 &        1 &   1 &   1 &   1 &       44 &  11 &  11 &  30 \\
    1 &       37 &  12 &  30 &  35 &        6 &  12 &  30 &  31 &       57 &  32 &  50 &  55 \\
    2 &       28 &  29 &   8 &   2 &        7 &  29 &   8 &   2 &       33 &  34 &  13 &   7 \\
    3 &       30 &  35 &  14 &   8 &       12 &  35 &  14 &   8 &       35 &  40 &  19 &  13 \\
    4 &       32 &  46 &  20 &  14 &       17 &  46 &  20 &  14 &       37 &  46 &  25 &  19 \\
                \hline
        \end{tabular}
\end{center}
\captionof{table}{Mediciones sobre un solo core
                  usando el lote de tareas de la figura \ref{fig:ej8-t1}}

Con estos valores podemos calcular el promedio de cada métrica para cada scheduler.

\begin{center}
        \begin{tabular}{| l | c | c | c |}
                \hline
    Scheduler & Waiting-time & Latencia & Turnaround \\
                \hline
    RR   &    33.5 &  8.6 & 41.2 \\
    FIFO &    24.6 & 24.6 & 32.6 \\
    SJF  &    14.6 & 14.6 & 23.6 \\
    RSJF &    15.8 & 11.2 & 24.8 \\
                \hline
        \end{tabular}
\end{center}
\captionof{table}{Promedio de métricas sobre un solo core
                  usando el lote de tareas de la figura \ref{fig:ej8-t1}}

\subsection{Mediciones sobre dos núcleos}

\begin{figure}[H]
        \centering
        \includegraphics[width=\textwidth]{ej8-cpu-rr-m}
        \caption{Ejecución sobre dos cores del lote de la figura
            \ref{fig:ej8-t1} con el scheduler RoundRobin}
        \label{fig:ej8-cpu-rr-m}
\end{figure}

\begin{figure}[H]
        \centering
        \includegraphics[width=\textwidth]{ej8-cpu-fcfs-m}
        \caption{Ejecución sobre dos cores del lote de la figura
            \ref{fig:ej8-t1} con el scheduler FIFO}
        \label{fig:ej8-cpu-fcfs-m}
\end{figure}

\begin{figure}[H]
        \centering
        \includegraphics[width=\textwidth]{ej8-cpu-sjf-m}
        \caption{Ejecución sobre dos cores del lote de la figura
            \ref{fig:ej8-t1} con el scheduler SJF}
        \label{fig:ej8-cpu-sjf-m}
\end{figure}

\begin{figure}[H]
        \centering
        \includegraphics[width=\textwidth]{ej8-cpu-rsjf-m}
        \caption{Ejecución sobre dos cores del lote de la figura
            \ref{fig:ej8-t1} con el scheduler RSJF}
        \label{fig:ej8-cpu-rsjf-m}
\end{figure}

\begin{center}
        \begin{tabular}{| c | c | c | c | c | c | c | c | c | c | c | c | c |}
                \hline
    Task & \multicolumn{4}{c |}{Waiting-time} & \multicolumn{4}{c |}{Latencia} & \multicolumn{4}{c |}{Turnaround} \\
                \cline{2-13}
          & RR & FIFO & SJF & RSJF & RR & FIFO & SJF & RSJF & RR & FIFO & SJF & RIFO \\
                \hline
    0 &       16 &   1 &   1 &  10 &        1 &   1 &   1 &   1 &       26 &  11 &  11 &  20 \\
    1 &       17 &   1 &   1 &  16 &        1 &   1 &   1 &   1 &       37 &  21 &  21 &  36 \\
    2 &       12 &   8 &   8 &   2 &        2 &   8 &   8 &   2 &       17 &  13 &  13 &   7 \\
    3 &       14 &  14 &  14 &   2 &        2 &  14 &  14 &   2 &       19 &  19 &  19 &   7 \\
    4 &       15 &  18 &  18 &   8 &        7 &  18 &  18 &   8 &       20 &  23 &  23 &  13 \\
                \hline
        \end{tabular}
\end{center}
\captionof{table}{Mediciones sobre dos cores
                  usando el lote de tareas de la figura \ref{fig:ej8-t1}}

Con estos valores podemos calcular el promedio de cada métrica para cada scheduler.

\begin{center}
        \begin{tabular}{| l | c | c | c |}
                \hline
    Scheduler & Waiting-time & Latencia & Turnaround \\
                \hline
    RR   &    14.8 &  2.6 & 23.8 \\
    FIFO &     8.4 &  8.4 & 17.4 \\
    SJF  &     8.4 &  8.4 & 17.4 \\
    RSJF &     7.6 &  2.8 & 16.6 \\
                \hline
        \end{tabular}
\end{center}
\captionof{table}{Promedio de métricas sobre dos cores
                  usando el lote de tareas de la figura \ref{fig:ej8-t1}}

\subsection{Conclusiones}

TODO


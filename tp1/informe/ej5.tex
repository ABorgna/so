\section{Ejercicio 5}

Dado el mismo lote de tareas que usamos para mediciones en el \hyperref[sec:ej2]{ejercicio 2}, vamos a calcular $latencia$, $waiting$-$time$ y $tiempo\ total$ de ejecución para cada proceso.

La $latencia$ de cada proceso es el tiempo desde que se termina de cargar hasta que comienza a ejecutarse.
El $waiting$ $time$ se calcula como el tiempo durante el que estuvo en estado $Ready$
y el $tiempo\ total$ se cuenta desde que se carga hasta que termina de ejecutarse.
\\

\begin{center}
\fbox{\parbox{6cm}
	{\noindent
		TaskCPU 10					\\
		@5:								 \\
		TaskConsola 5 1 4	 \\
		@6:								 \\
		TaskConsola 5 1 2	 \\
		@8:								 \\
		TaskCPU 10
	}
}
\end{center}
\captionof{figure}{Lote de tareas usado}

\begin{figure}[H]
		\centering
		\includegraphics[width=\textwidth]{ej5-2}
		\caption{Ejecución con un único core, 2 ciclos para cada cambio de contexto y 2 quantums}
		\label{fig:ej5.2}
\end{figure}

\begin{center}
		\begin{tabular}{| l | l | l | l |}
				\hline
				Tarea & Waiting-time & Latencia & Tiempo Total	\\ \hline
				0 &	38	 &	2	 &	48	\\
				1 &	53	 &	3	 &	75	\\
				2 &	62	 &	5	 &	78	\\
				3 &	58	 &	10 &	68	\\
				\hline
		\end{tabular}
\end{center}
\captionof{table}{: Correspondiente a la figura \ref{fig:ej5.2} }

\begin{figure}[H]
		\centering
		\includegraphics[width=\textwidth]{ej5-5}
		\caption{Ejecución con un único cores, 2 ciclos para cada cambio de contexto y 5 quantums}
		\label{fig:ej5-5}
\end{figure}

\begin{center}
		\begin{tabular}{| l | l | l | l |}
				\hline
				Tarea & Waiting-time & Latencia & Tiempo Total	\\ \hline
				0 & 10 & 2  & 20 \\
				1 & 44 & 4  & 65 \\
				2 & 52 & 6  & 68 \\
				3 & 24 & 14 & 34 \\
				\hline
		\end{tabular}
\end{center}
\captionof{table}{: Correspondiente a la figura \ref{fig:ej5-5} }


\begin{figure}[H]
		\centering
		\includegraphics[width=\textwidth]{ej5-10}
		\caption{Ejecución con un único cores, 2 ciclos para cada cambio de contexto y 10 quantums}
		\label{fig:ej5-10}
\end{figure}

\begin{center}
		\begin{tabular}{| l | l | l | l |}
				\hline
				Tarea & Waiting-time & Latencia & Tiempo Total	\\ \hline
				0     &  2           &  2       & 12            \\
				1     &  39          &  9       & 61            \\
				2     &  48          &  11      & 64            \\
				3     &  12          &  12      & 22            \\
				\hline
		\end{tabular}
\end{center}
\captionof{table}{: Correspondiente a la figura \ref{fig:ej5-10} }

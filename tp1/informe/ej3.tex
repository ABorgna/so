\section{Ejercicio 3}

La \textbf{TaskBatch} debe usar la CPU durante $total\_cpu$ ciclos de reloj.
Como cada bloqueo utiliza un ciclo para iniciarse, el tiempo extra que tendremos que
usar la CPU sera de $t_{total} = total\_cpu - cant\_bloqueos$.

Para acomodar todos los bloqueos pseudo-aleatoriamente decidimos en qué luego de cuanto
tiempo extra lanzar cada una utilizando la función \textsf{rand()} y la insertamos en un vector ordenado.
Luego procedemos recorremos el array esperando el tiempo necesario.

El algoritmo se detalla en el listing \ref{ej3-algo}

\begin{lstlisting} [caption={Generacion aleatoria de la secuencia de bloqueos},label=ej3-algo]j
for $i$ in range($cant\_bloqueos$):
    agregar (rand() % ($t_{total} + 1$)) al array ordenado $bloqueos$

$tiempoExtraConsumido$ $\leftarrow$ 0
for $t_{bloqueo}$ in $bloqueos$:
    if $tiempoExtraConsumido$ < $t_{bloqueo}$:
        usar la cpu por ($t_{bloqueo}$ - $tiempoExtraConsumido$) ciclos
        $tiempoExtraConsumido$ $\leftarrow$ $t_{bloqueo}$
    bloquear la tarea por 2 ciclos

if $tiempoExtraConsumido$ < $t_{total}$:
    usar la cpu por ($t_{total}$ - $tiempoExtraConsumido$) ciclos

\end{lstlisting}

Utilizando un lote de \textbf{TaskBatch} con parámetros detallados en la tabla \ref{table:ej3-lote}
podemos ver en la figura \ref{fig:ej3} como se logra el comportamiento deseado.

\begin{center}
    \begin{tabular}{| l | l | l |}
        \hline
        Tarea & total\_cpu & cant\_bloqueos \\ \hline
        0 & 10 & 4 \\
        1 & 15 & 4 \\
        2 & 10 & 2 \\
        3 & 15 & 6 \\
        \hline
    \end{tabular}
    \label{table:ej3-lote}
\end{center}

\begin{figure}[H]
    \centering
    \includegraphics[width=\textwidth]{ej3}
    \caption{Lote de 4 TaskBatch}
    \label{fig:ej3}
\end{figure}


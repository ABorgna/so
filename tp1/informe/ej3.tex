\section{Ejercicio 3}

La \textbf{TaskBatch} debe usar la CPU durante $total\_cpu$ ciclos de reloj.
Como cada bloqueo utiliza un ciclo para iniciarse, podemos distribuirlos uniformemente
eligiendo para cada uno en qué tiempo de ejecución correr, checkeando que no se repita.
Como no se puede bloquear en el último tiempo de cpu del proceso, generamos las
posiciones como un número aleatorio entre $0$ y $total\_cpu - 1$ con la función
\textsf{rand()}.

El algoritmo se detalla en el listing \ref{ej3-algo}

\begin{lstlisting} [caption={Generacion aleatoria de la secuencia de bloqueos},label=ej3-algo]j
set $bloqueos$
for $i$ in range($cant\_bloqueos$):
    do:
        $tiempo$ $\gets$ rand() % ($total_{cpu}$ - 1)
    while $tiempo$ ya esta en $bloqueos$

    agregar $tiempo$ al conjunto $bloqueos$

$tiempoConsumido$ $\gets$ 0
for $t_{bloqueo}$ in $bloqueos$: // $bloqueos$ se recorre ordenado
    if $tiempoConsumido$ < $t_{bloqueo}$:
        usar la cpu por ($t_{bloqueo}$ - $tiempoConsumido$) ciclos
        $tiempoConsumido$ $\gets$ $t_{bloqueo}$
    bloquear la tarea por 2 ciclos
    $tiempoConsumido$++

if $tiempoConsumido$ < $total_{cpu}$ - 1:
    usar la cpu por ($total_{cpu}$ - 1 - $tiempoConsumido$) ciclos

\end{lstlisting}

Utilizando un lote de \textbf{TaskBatch} con parámetros detallados en la tabla \ref{table:ej3-lote}
podemos ver en la figura \ref{fig:ej3} como se logra el comportamiento deseado.

\begin{center}
    \begin{tabular}{| l | l | l |}
        \hline
        Tarea & total\_cpu & cant\_bloqueos \\ \hline
        0 & 10 & 4 \\
        1 & 15 & 4 \\
        2 & 10 & 2 \\
        3 & 15 & 6 \\
        \hline
    \end{tabular}
    \label{table:ej3-lote}
\end{center}

\begin{figure}[H]
    \centering
    \includegraphics[width=\textwidth]{ej3}
    \caption{Lote de 4 TaskBatch}
    \label{fig:ej3}
\end{figure}

